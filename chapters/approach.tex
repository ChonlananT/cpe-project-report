\chapter{\ifproject%
\ifenglish Project Structure and Methodology\else โครงสร้างและขั้นตอนการทำงาน\fi
\else%
\ifenglish Project Structure\else โครงสร้างของโครงงาน\fi
\fi
}

\makeatletter

% \renewcommand\section{\@startsection {section}{1}{\z@}%
%                                    {13.5ex \@plus -1ex \@minus -.2ex}%
%                                    {2.3ex \@plus.2ex}%
%                                    {\normalfont\large\bfseries}}

\makeatother
%\vspace{2ex}
% \titleformat{\section}{\normalfont\bfseries}{\thesection}{1em}{}
% \titlespacing*{\section}{0pt}{10ex}{0pt}

\section{การออกแบบ}

\subsection{Ansible}
\hspace{0.5in} Ansible เป็น Open Source Software Automation Tool ที่ใช้ YAML syntax ในการเขียน Playbook ข้อดีของการใช้ Ansible ในการออกแบบมีดังนี้ 
\begin{itemize}
  \item ใช้งานง่าย: Ansible เขียนด้วย YAML syntax เข้าใจง่าย
  \item Agentless: ไม่จำเป็นต้องติดตั้ง agent บนเครื่อง Remote
  \item Powerful: รองรับการ Automation งานต่างๆ มากมาย
  \item Flexible: รองรับระบบปฏิบัติการหลายประเภท
\end{itemize}

\subsection{AWX}
\hspace{0.5in} AWX เป็น Open Source Web Application ที่ช่วยจัดการ Workflow ของ Ansible ได้อย่างมีประสิทธิภาพ ข้อดีของการใช้ AWX ในการจัดการอุปกรณ์เน็ตเวิร์ค มีดังนี้
\begin{itemize}
  \item Centralized Management: ควบคุมจัดการ Jobs, Playbooks, Inventory และ Credentials ทั้งหมดจากจุดเดียว
  \item Role-Based Access Control: กำหนดสิทธิ์การเข้าถึงสำหรับผู้ใช้แต่ละคน
  \item Integrations: รองรับการผสานรวมกับ Tools อื่นๆ เช่น Jenkins, GitLab และ GitHub
  \item Notification: มีการแจ้งเตือนผู้ใช้เมื่อมี Errors เกิดขึ้น
\end{itemize}
\hspace{0.5in} การใช้ AWX ใน Web Application สำหรับ Automation จะช่วยติดตามสถานะการทำงาน จัดการกลุ่มเป้าหมาย (Inventory) ค้นหาและดูข้อมูลย้อนหลัง และสามารถตรวจสอบปัญหาต่างๆที่เกิดขึ้นในเมื่อเกิด Errors

\subsection{Web Application}
\hspace{0.5in} เนื่องจากการใช้งาน Ansible AWX นั้นมีข้อดีที่สามารถเข้าไปจัดการอุปกรณ์ได้ทีละหลายอุปกรณ์ แต่ยังมีข้อจำกัดคือ ไม่สามารถเช็คความถูกต้องเกี่ยวกับ .....(ว่านช่วยด้วยยยยย)..... จากการรัน Templates บน Ansible AWX จึงออกแบบ Web Application ที่สามารถสร้าง Ansible Playbooks และสามารถใช้ AWX API ในการดึงข้อมูลการจัดการอุปกรณ์เน็ตเวิร์ค ต่างๆที่อยู่ใน Ansible AWX มาแสดงใน Web Application และสามารถนำคำสั่งที่เขียนใน Ansible Playbooks มาตรวจสอบความถูกต้องว่าคำสั่งที่สั่ง ครบหรือไม่ หรือ เช็คว่าการทำงานของอุปกรณ์ เน็ตเวิร์คทำงานถูกต้องหรือไม่

\section{โครงสร้างการทำงาน}