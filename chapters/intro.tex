\chapter{\ifenglish Introduction\else บทนำ\fi}

\section{\ifenglish Project rationale\else ที่มาของโครงงาน\fi}
\hspace{0.5in}
การตั้งค่าและการตรวจสอบความถูกต้องอุปกรณ์เครือข่ายในปัจจุบันนั้นสามารถทำได้ยากสำหรับบุคคลทั่วไปที่ไม่มีความเชี่ยวชาญในด้านนี้ เนื่องจากผู้ที่ต้องทำการตั้งค่าต้องมีความรู้พื้นฐานด้านชุดคำสั่งสำหรับการตั้งค่าอุปกรณ์เครือข่ายและต้องทำการตั้งค่าและตรวจสอบผ่าน Command Line Interfaces (CLIs) ซึ่งการตั้งค่าอุปกรณ์เครือข่ายในแต่ละรอบนั้นสามารถทำได้ทีละเครื่องเท่านั้น หากมีกรณีที่ผู้ดูแลระบบต้องทำการตั้งค่าอุปกรณ์หลายๆ อุปกรณ์โดยต้องใช้ชุดคำสั่งเดิมจะทำให้เกิดการทำงานที่ซ้ำซ้อนและทำให้เสียเวลาได้และในบางครั้งผู้ดูแลระบบมีโอกาสที่จะตั้งค่าอุปกรณ์เครือข่ายผิดจากการทำงานด้วยมือโดยที่ไม่ได้ตั้งใจอีกด้วย
ถึงแม้ว่าทำการตั้งค่าอุปกรณ์เครือข่ายเสร็จแล้วผู้ดูแลยังต้องทำการตรวจสอบการทำงานของอุปกรณ์เครือข่ายว่าถูกต้องหรือไม่ ซึ่งการตรวจสอบการทำงานของอุปกรณ์เครือข่ายนั้นผู้ดูแลต้องมีความรู้รวมถึงความเชี่ยวชาญในการวิเคราะห์การทำงานของอุปกรณ์เครือข่ายเนื่องจากผู้ดูแลจำเป็นต้องรู้ว่าต้องใช้ชุดคำสั่งใดในการตรวจสอบความถูกต้องและต้องรู้ว่าควรตรวจสอบข้อมูลส่วนไหนจาก output ที่ได้มา

\hspace{0.5in}
ในปัจจุบันมี tools สำหรับการ Automation ชื่อว่า Ansible ซึ่งเป็นเครื่องมือที่สามารถ SSH เข้าไปจัดการตั้งค่าอุปกรณ์เครือข่ายหลายๆเครื่องในครั้งเดียวได้ 
\section{\ifenglish Objectives\else วัตถุประสงค์ของโครงงาน\fi}
\begin{enumerate}
    \item
\end{enumerate}

\section{\ifenglish Project scope\else ขอบเขตของโครงงาน\fi}

\subsection{\ifenglish Hardware scope\else ขอบเขตด้านฮาร์ดแวร์\fi}

\subsection{\ifenglish Software scope\else ขอบเขตด้านซอฟต์แวร์\fi}

\section{\ifenglish Expected outcomes\else ประโยชน์ที่ได้รับ\fi}

\section{\ifenglish Technology and tools\else เทคโนโลยีและเครื่องมือที่ใช้\fi}

\subsection{\ifenglish Hardware technology\else เทคโนโลยีด้านฮาร์ดแวร์\fi}

\subsection{\ifenglish Software technology\else เทคโนโลยีด้านซอฟต์แวร์\fi}

\section{\ifenglish Project plan\else แผนการดำเนินงาน\fi}

\begin{plan}{11}{2023}{2}{2024}
    \planitem{11}{2023}{11}{2023}{ศึกษาชุดคำสั่งการตั้งค่าและการตรวจสอบของ Cisco}
    \planitem{12}{2023}{1}{2024}{ศึกษาและทดลองใช้งาน Ansible}
    \planitem{1}{2024}{2}{2024}{ศึกษาและทดลองใช้งาน Ansible AWX}
    \planitem{2}{2024}{2}{2024}{เขียนรายงาน}
\end{plan}

\section{\ifenglish Roles and responsibilities\else บทบาทและความรับผิดชอบ\fi}
อธิบายว่าในการทำงาน นศ. มีการกำหนดบทบาทและแบ่งหน้าที่งานอย่างไรในการทำงาน จำเป็นต้องใช้ความรู้ใดในการทำงานบ้าง

\section{\ifenglish%
Impacts of this project on society, health, safety, legal, and cultural issues
\else%
ผลกระทบด้านสังคม สุขภาพ ความปลอดภัย กฎหมาย และวัฒนธรรม
\fi}

แนวทางและโยชน์ในการประยุกต์ใช้งานโครงงานกับงานในด้านอื่นๆ รวมถึงผลกระทบในด้านสังคมและสิ่งแวดล้อมจากการใช้ความรู้ทางวิศวกรรมที่ได้
