\chapter{\ifenglish Conclusions and Discussions\else บทสรุปและข้อเสนอแนะ\fi}

\section{\ifenglish Conclusions\else สรุปผล\fi}

\hspace{0.5in}โครงงานนี้ได้จัดทำแอพพลิเคชั่นตั้งค่าและตรวจสอบอุปกรณ์เครือข่ายอัตโนมัติโดยใช้แอนสิเบิล
สำหรับบุคคลทั่วไป หรือผู้ดูแลระบบที่ไม่ชำนาญการใช้ชุดคำสั่งแบบ CLIs โดยใช้ Ansible AWX เพื่อ
รับข้อมูลและสั่งการให้อุปกรณ์เครือข่ายทำงานตามที่ต้องการ จากการทดสอบระบบพบว่าระบบ
สามารถทำงานได้ถูกต้องแต่มีข้อจำกัดคือการเข้าถึงอุปกรณ์แบบ offline

\section{\ifenglish Challenges\else ปัญหาที่พบและแนวทางการแก้ไข\fi}

\hspace{0.5in}ในการทำโครงงานนี้ พบว่าเกิดปัญหาหลักๆ ดังนี้
\begin{enumerate}
    \item ในการเรียกใช้งาน Templates จำเป็นต้องเรียกใช้งานแบบ online เท่านั้น ไม่สามารถเชื่อต่อกับอุปกรณ์เป้าหมายแบบ offline ได้
    \item อะไรอีก
  \end{enumerate}
\section{\ifenglish%
Suggestions and further improvements
\else%
ข้อเสนอแนะและแนวทางการพัฒนาต่อ
\fi
}

\hspace{0.5in}ข้อเสนอแนะเพื่อพัฒนาโครงงานนี้ต่อไป มีดังนี้

\hspace{0.5in}\begin{enumerate}
    \item พัฒนาให้รองรับการใช้งานแบบ offline โดยให้ผู้ใช้สามารถดาวน์โหลดและเรียกใช้งาน Templates ได้โดยไม่ต้องเชื่อมต่อกับอินเทอร์เน็ต

    \hspace{0.5in}เพื่อเพิ่มความสะดวกสบายในการใช้งาน การพัฒนาให้รองรับการใช้งานแบบ offline จะช่วยให้ผู้ใช้สามารถดาวน์โหลด Templates และใช้งานโครงงานได้ทุกที่ โดยไม่จำเป็นต้องมีการเชื่อมต่อกับอินเทอร์เน็ตตลอดเวลา

    \item อะไรอีก

    \hspace{0.5in}\item อาจพิจารณาการเพิ่มความหลากหลายใน Templates หรือฟีเจอร์เพิ่มเติมที่ทำให้โครงงานมีประสิทธิภาพมากยิ่งขึ้น ตลอดจนการทำให้ตัวโครงงานมีการใช้งานที่หลากหลายและตอบสนองต่อความต้องการของผู้ใช้มากขึ้น
\end{enumerate}
